%%----------------------------------------------------------------------
%%----------------------------------------------------------------------
\missiontitle{Mission 2: Those Big Gunz Are Ours!}

\teaser{}

%%----------------------------------------------
\begin{tablesetup}
  \dawnofwar

  \bigskip%
  After determining deployment zones, place two primary objectives
  markers each 18'' from a short table edge and 24'' from the long
  edges.  These are designated as command center markers, explained
  below.  The teams then roll off on~D6 and in that order alternate
  placing a total of four more objective markers.  Each team must
  place their first marker in their opponent's deployment zone and
  their second in their own deployment zone.  Label all of the
  markers, including the command centers,~1 through~6 in any fashion.


\end{tablesetup}

%%----------------------------------------------
\begin{missionrules}

\nightfighting

\missionsubheading{Orbital Strike.}%
If a team controls one or both command center markers at the beginning
of their movement phase, they gain a special shooting attack for that
turn.  Normal rules for holding markers apply, except that for the
purposes of Orbital Strike alone the markers cannot be controlled by
Beasts, Swarms, Vehicles, or Monstrous and Gargantuan Creatures.  Such
units may still score and contest these markers as usual.

\bigskip%
If a team controls one marker, at the beginning of the shooting phase
a model within 3'' of the marker (applying the unit type restrictions
above, but not necessarily a model in range at the start of movement),
may fire with the following profile instead of any of its weapons:
\begin{squishitemize}
\item Range Unlimited, S6, AP5, Ignore Cover, Blast, Barrage
\end{squishitemize}

\bigskip%
If a team controls both markers, at the beginning of the shooting
phase one model within 3'' of either marker (applying the unit type
restrictions above, but not necessarily a model in range at the start
of movement), may fire with either of the following profiles instead
of any of its weapons:
\begin{squishitemize}
\item Range Unlimited, S6, AP5, Ignore Cover, Large Blast, Barrage
\item Range Unlimited, S6, AP3, Ignore Cover, Blast, Barrage
\end{squishitemize}

\bigskip%
Any of these special attacks may target a different unit than the rest
of the shooter's unit, but do not count for purposes of assault charge
eligibility.

\missionsubheading{Maelstrom.}%
At the start of each team turn, the active team does the following
until it has as many tactical objectives in play as the current turn
number: Roll a~D66 and consult your tactical objective table, attached
at the end of this mission packet.  If that objective is already in
play for you, has been achieved, or is scratched off, roll again.
Similarly, if that objective would be provably impossible to score,
e.g., your opponent has no characters remaining, roll again.  Once a
valid objective has been rolled, mark it as in play.

Targets cannot be nominated or chosen for a tactical objective marked
with a $\dagger$ that have already been chosen for a $\dagger$
objective you have in play.

\smallskip%
At the end of your turns, check the requirements for each tactical
objective you have in play.  For each fixed-value objective met, mark
it as achieved and score the associated value in \emph{mission points}
(n.b.: not \emph{victory points}).  Tactical objectives with a value
of X may be kept in play as long as you wish.  At the end of any of
your turns while in play they may be marked as achieved and scored as
indicated.  Once achieved, objectives are no longer considered in play
and cannot be put in play or scored again.

Multiple objectives can be scored in a turn, caveat that you cannot
achieve multiple tactical objectives with the same exact title in the
same turn using the same marker(s) or unit(s).  E.g., to score both
Storm objectives at once, you would need to simultaneously control two
separate markers in the enemy deployment zone.

At the end of your turn you may scratch out one of your tactical
objectives in play to remove it from play.

\smallskip%
Tactical objectives in play, achieved, and scratched out are not
secret.

\end{missionrules}


%%----------------------------------------------
\begin{scoring}
  
\begin{primaries}

At game end, compare mission points earned through tactical
objectives achieved and award victory points to the higher and lower
scorer as follows:

%\definecolor{Gray}{gray}{0.9}
%\definecolor{DGray}{gray}{0.75}
\newcolumntype{a}{>{\columncolor{gray!25}}c}
\newcolumntype{b}{>{\columncolor{gray!50}}r}
\bigskip\centerline{\setlength{\tabcolsep}{12pt}%
\begin{tabular}{|b|c|a|c|a|c|a|}
\hline
{\bf Difference}     & 0     & 1--2  & 3--4  & 5--6  & 7     & 8+\\
{\bf Victory Points} & 4 / 4 & 5 / 4 & 6 / 3 & 7 / 2 & 8 / 1 & 9 / 0\\
\hline
\end{tabular}}


\end{primaries}

\begin{secondaries}
  \breaktheirback

  \holdthefield

  \assassination
\end{secondaries}

\end{scoring}
